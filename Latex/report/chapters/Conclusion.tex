%_____Zusammenfassung, Ausblick_________________________________
\chapter{Conclusion}
We quantified the information loss accompanying the resolution drop of IMU sensors in Parkinson's disease classification by calculating the difference in conditional entropy between different resolutions after the KNN classification. We resampled the data at sampling frequencies of 25 Hz, 10 Hz, and 5 Hz to simulate low-resolution IMU sensors, respectively, and the results showed that the information loss was less for the 50 Hz drop to 25 Hz and more significant for the 50 Hz drop to 10 and 5 Hz. However, limited by the drawbacks of the KNN classifier, if our dataset is unbalanced, we can increase the weights and then use KNN classification again. We can continue to use different classifiers to quantify information loss in the future. For example, other classifiers can be used only for two sample classification categories.
On the other hand, the data of Parkinson's disease patients tend to be large, so KNN classification still has advantages when dealing with large amounts of data. The larger the sample size of the Monte Carlo method, the closer the approximation is to the actual value. Similarly, we can continue to investigate the effect of the choice of k parameters on the information loss in KNN classification. In summary, we conclude that IMU sensors as low as 25 Hz have less information loss and can interpret the results of these works as low-resolution IMU sensors that still work well between 25 and 50 Hz. This could be a promising aspect of using low-cost sensors to study Parkinson's disease.
